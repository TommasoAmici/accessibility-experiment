\documentclass[12pt, a4paper]{article}

\usepackage{graphicx}
\usepackage{setspace}
\usepackage{tikz}
\usetikzlibrary{positioning}

% STYLE
\usepackage[pdfborder={0 0 0}]{hyperref}
\usepackage{geometry}
\usepackage{numprint}

% BIBLIOGRAPHY
\usepackage{apacite} % after hyperref

\usepackage{epigraph}
\setlength\epigraphwidth{.8\textwidth}

% HYPOTHESES
% \usepackage{etoolbox,changepage}
\usepackage{ntheorem}
\newtheorem{req}{Research question}
\newtheorem{hyp}{Hypothesis}
% \AtBeginEnvironment{hyp}{\begin{adjustwidth}{3em}{3em}}
% \AtEndEnvironment{hyp}{\end{adjustwidth}}
\theoremseparator{:}

\title{The impact of web accessibility on the desire for avoidance}
\author{Tommaso Amici}

\begin{document}
\doublespacing
\pagenumbering{gobble}

\makeatletter
\begin{titlepage}
  \begin{center}
    \includegraphics[scale=0.5]{images/uva_logo.eps}

    \vspace{0.8cm}

    \huge{\textbf{\@title}}

    \vspace{0.8cm}

    \large{\@author}

    \normalsize{\today}

  \end{center}

  \vfill

  % \includegraphics[width=0.4\textwidth]{university}

  \normalsize

  Student number: 12306444

  MSc Business Administration - Digital Marketing

  Amsterdam Business School, University of Amsterdam

  Supervisor: Joris Demmers

  EBEC approval number: \todo{Guidelines 4.4.1 EBEC approval}

\end{titlepage}
\makeatother


\newpage

\pagenumbering{roman}

\begin{abstract}
  \#TODO
\end{abstract}
\newpage

\tableofcontents
\listoffigures
\listoftables
\newpage

\pagenumbering{arabic}

\section{Introduction}

\epigraph{The power of the Web is in its universality. Access by everyone regardless
  of disability is an essential aspect.}{Tim Berners-Lee, W3C Director and inventor
  of the World Wide Web}

\epigraph{Accessibility is the right thing to do.}{\citeA{valdes1999} for the United Nations}

\noindent
After more than twenty years, we have reached a tipping point: accessibility is not only
the right thing to do, but it is a legal requirement for public sector bodies in the EU
since the approval of the Web Accessibility Directive in 2016, and it may soon be a legal
requirement for private sector bodies should the European Accessibility Act be approved.

In \citeA{carter2001web}, a spokesperson for an electronics retailer when asked about
Internet users with disabilities is quoted as saying: ``that's not a market we've
thought about pursuing'' (p. 227). However, looking at some recent statistics paints a
striking picture: the Office for National Statistics estimates that, in 2020, 22\% of
internet users in the UK have self-assessed disabilities
\citeyear{ukinternetusers2020}, and according to different estimates 100 million people
in the EU and one billion people in the world have some form of disability
\cite{euWebAccessibility2021, whoDisability2021}.

For online businesses that target the general population, failing to adapt their
website to the needs of users with disabilities means potentially missing out on
revenue from this group of potential clients. Moreover, this failure may have legal
implications. However, assuming they are managed rationally, businesses may be
operating under the impression that web accessibility does not affect purchase
intention, and as a consequence they will not dedicate resources to it.

In this paper, we will attempt to answer the following questions:

\begin{req}
  Is accessibility a concern for online businesses, or are they justified in ignoring it,
  from an economic perspective?
\end{req}

The next section explains what web accessibility is, and why many websites fail to
properly support disabled users.

\section{Theoretical background and hypotheses}

\subsection{Web accessibility}\label{sec:web-accessibility}

According to the Web Content Accessibility Guidelines (WCAG) 2.1, the objective of
developers and designers implementing accessible websites is to make content ``more
accessible to a wider range of people with disabilities, including accommodations for
blindness and low vision, deafness and hearing loss, limited movement, speech
disabilities, photosensitivity, and combinations of these, and some accommodation for
learning disabilities and cognitive limitations'' \cite[para. 1]{wcag21}.

It is important to note that disability can be temporary or situational: for example,
people with a broken arm are temporarily disabled; while a parent holding a child is
situationally disabled, i.e., they cannot use two arms in that specific situation
\cite{henry2006}. This means that, effectively, any Internet user is potentially
disabled in any given moment. The respondents of a survey conducted in 2015 among web
accessibility specialists agreed that accessibility is applicable to everyone and that
it is highly related to usability \cite{yesilada2015}.

However, another survey conducted in 2018 among Brazilian web developers illustrates
how these topics are still not fully understood by professionals: of the 404
respondents, the majority had never developed an accessible website, and 33.2\% did not
plan on integrating accessibility best practices in future projects
\cite{antonelli2018}.

\citeA{carter2001web} nonetheless reports a second reason why accessibility is often
ignored by web developers and businesses: ``making the site accessible makes it
boring'' (p. 227). Designers are often convinced that building an accessible
site results in uninteresting and plain sites. Accessibility is thus considered useless
for non-disabled users, if not outright damaging to their experience by forcing them to
use a subpar product. On the contrary, \citeA{schmutz2016implementing} finds that
non-disabled users also benefit from a more accessible interface, and research by
\citeA{petrie2004} finds that accessibility does not constrain visual design.

The existing literature focuses primarily on the perceptions of businesses,
practitioners, and users, both disabled and non-disabled, highlighting how
accessibility is often ignored while it should not, both from a moral perspective and
from a practical perspective. However, a question that is not directly answered is
whether a lack of accessibility can lead to a loss of customers for an online business.

\citeA{morris2016} lightly touches upon this topic: analyzing the results of a survey
conducted on blind users of Twitter, they find that, as the platform becomes less
accessible with more and more users posting visual content rather than written content,
blind users' desire for avoidance increases because they simply cannot use the site in a
meaningful way anymore. On this note, it is worth noting that visually disabled users do
not appreciate having a separate text-only copy of a website out of fear that it may not
be as updated as the main site \cite{weeratunga2015}.

\subsection{Internet usage}\label{sec:internet-usage}

\section{Research method}

\section{Results}

\section{Discussion}
\subsection{Theoretical contributions}
\subsection{Managerial implications}
\subsection{Limitations and future research}

\newpage
\bibliographystyle{apacite}
\bibliography{references}

\end{document}
