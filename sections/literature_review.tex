\section{Literature review}

According to the Web Content Accessibility Guidelines (WCAG) 2.1, the objective of
developers and designers implementing accessible websites is to make content ``more
accessible to a wider range of people with disabilities, including accommodations for
blindness and low vision, deafness and hearing loss, limited movement, speech
disabilities, photosensitivity, and combinations of these, and some accommodation for
learning disabilities and cognitive limitations; but will not address every user need
for people with these disabilities'' \cite{wcag21}.

It is important to note that disability can be temporary or situational: for example
people with a broken arm are temporarily disabled, while a parent holding a child is
situationally disabled in that they cannot use two arms in that specific situation
\cite{henry2006}. This means that effectively any Internet user is potentially disabled
in a given moment. This is also supported by a survey conducted in 2015 among web
accessibility specialists where respondents agreed that accessibility is applicable to
everyone and that it is highly related to usability \cite{yesilada2015}.

\citeA{carter2001web} reports a second reason why accessibility is often
ignored by web developers and businesses: ``making the site accessible makes it
boring''. Moreover, designers are convinced that building an accessible site results in
uninteresting and plain sites. Accessibility is thus considered useless for
non-disabled users, if not outright damaging to their experience, forcing them to use a
subpar product. On the contrary, non-disabled users also benefit from a more accessible
interface, and accessibility does not constrain visual design
\cite{schmutz2016implementing, petrie2004}.

The literature focuses primarily on the perceptions of businesses, practitioners, and
users, both disabled and non-disabled, highlighting how accessibility is often ignored
while it should not, both from a moral perspective and from a practical perspective.
However, a question that is not directly answered is whether a lack of accessibility
can lead to a loss of customers for an online business.

When it comes to disabled users, the literature on electronic word of mouth (eWOM)
finds they rely on online reviews to bypass physical requirements, but a recent
empirical study finds that, in the hotel industry, disabled guests' perception of risk
is positively mitigated by trustworthy reviews, however it finds no significant effect
on purchase decision \cite{williams2006, zhang2021}.

In general, users that complain online tend to hold a grudge and although their desire
for revenge diminishes in time, their desire for avoidance does not and they will take
their business elsewhere \cite{tripp2011unhappy}.

In this paper, data collected from social media will be analyzed to determine whether
users unhappy with the accessibility of a website make online complaints about it, and
whether this is a concern for businesses or not.

Previous studies have shown empirically that online reviews are dominated by activists,
users who post often and who tend to be negative \cite{moe2012}. Our proposed model
takes this into account, and recurring complaints from the same users will be flagged
as such, to avoid magnifying the complaints of a loud minority.
