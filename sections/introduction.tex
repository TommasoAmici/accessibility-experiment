\section{Introduction}

\epigraph{The power of the Web is in its universality. Access by everyone regardless
  of disability is an essential aspect.}{Tim Berners-Lee, W3C Director and inventor of the World Wide Web}

\epigraph{Accessibility is the right thing to do.}{\citeA{valdes1999} for the United Nations}

\noindent
After more than twenty years, we have reached a tipping point: accessibility is not only
the right thing to do, but it is a legal requirement for public sector bodies in the EU
since the approval of the Web Accessibility Directive in 2016, and it may soon be a legal
requirement for private sector bodies should the European Accessibility Act be approved.

In \citeA{carter2001web}, a spokesperson for an electronics retailer when asked about
Internet users with disabilities is quoted as saying: ``that's not a market we've
thought about pursuing.'' However, looking at some recent statistics paints a different
picture: the Office for National Statistics estimates that, in 2020, 22\% of internet
users in the UK have self-assessed disabilities \citeyear{ukinternetusers2020}, and
according to different estimates 100 million people in the EU and one billion people in
the world have some form of disability \cite{euWebAccessibility2021,
  whoDisability2021}. Voluntarily excluding such a huge part of your customer base is at
best shortsighted, and at worst illegal in the near future.

The next section explains what web accessibility is, and why many websites fail to
properly support disabled users.
